\section{Introduction}\label{sec:intro}

% P1: Motivation. 
% - What is the problem area you are working in and why is it important? 
% - Why is the problem of interest and importance to the larger community?
Recommender systems have become a hot research area in recent years and 
  many innovative approaches have been proposed both scientifically and commercially. 
Many recommender systems deal with books and daily commodities; 
  they need to be more efficient and accurate, and cover more fields.
Online trading is an overwhelming trend and E-commerce is a promising field. 
For many businesses, online opinion has turned into a kind of virtual currency 
  that can make or break a product in the marketplace \cite{Wright09}. 
There are so many items (products) in E-commerce systems that 
  picking an item is time-consuming for individuals.
Therefore recommender systems are key to the success of such E-commerce systems, 
  and shortening the time for an individual means improving the 
  throughput of the E-commerce system, thus contributing more transactions.

% P2: What is the specific problem considered in this paper? 
The soaring E-commerce industry demands recommender systems 
  with higher efficiency and accuracy, as well as support for more 
  personalized recommendation.
Personalized recommendation can be achieved by using data mining techniques on 
  massive amount of user profiles and behaviors (so-called collaboration) 
  regardless of the features of recommended items themselves. 
However, this kind of personalized recommendation is not applicable to 
  personalized apparel recommendation where the choices of people with similar interests 
  are sensitive to apparel particulars and may differ in case of only tiny differences. 
In this case, it is essential to capture as many features as possible 
  which are associated with both the clothes themselves and the user's preference 
  so that the personalized recommendation can be achieved in better performance. 

% P3: Summarize what are the main contributions of your paper.
% - What is the general approach taken? 
% - Why are the specific results significant?
In this paper, we propose a novel idea and framework for personalized apparel recommendation
  which combines both item images and textual labels by using sparse coding and 
  support vector machine (SVM). 
Sparse coding provides a class of algorithms for finding succinct representations of stimuli; 
  given only unlabeled input data, it learns basis functions that capture higher-level features 
  in the data \cite{Lee07}.
SVM is a widely used machine learning method for classification and regression analysis,
  based on the principle of structural risk minimization, which performs well when applied to 
  data outside the training set \cite{El02}.

% P4: What are the differences in what you are doing, and what others have done?
Given the images, the textual descriptions and the user ratings of the garments,
  our framework is capable of personalized recommendation according to various metrics
  such as style and diversity.
The novelty of our approach also comes from the fact that
  we try to not only employ a reasonable method to utilize 
  image information but also combine the image and text information together.
To the best of our knowledge, personalized apparel recommendation has not been 
  studied before in the context of E-commerce systems and the combination of 
  image feature extraction and machine learning.

% P5  
The remainder of this paper is structured as follows. 
Section \ref{sec:overview} presents an overview of our framework. In Section \ref{sec:challenge} several technical challenges in personalized apparel recommendation are discussed.
Section \ref{sec:approach} presents our approach for the novel framework for personalized apparel recommendation.
Section \ref{sec:eval} shows some preliminary experiment results under our framework.
Section \ref{sec:conclusion} concludes this paper and discusses some future work.
Section \ref{sec:related} discusses some of the related work.


