\section{Our Approach}\label{sec:approach}

The first step is to process text labels. A certain number of keywords 
  could be picked to describe clothing characteristics like 
  \emph{long sleeve} in sleeve length, 
  \emph{leisure} in styles and 
  \emph{cotton} in fabrics. 
Then a vector in binary is used to record such features for both clustering and feature selection.

In the next step, HAC (Hierarchical Agglomerative Clustering) was 
employed to reveal similarities among the text labels of different 
garments, then we manually set a threshold as the number of clusters to divide all the garments to several groups, which serve as the basis for future classification.

For images, at first they are preprocessed to gray-scale 
  pictures with minimum disturbance from the background 
  and the photo model, which is approached by morphological 
  manipulation using OpenCV. 
Then sparse coding is applied with respect to a learnt dictionary  
  to delineate pictures by feature vectors. 
Thus each image can be successfully represented and be ready for machine learning. 
It is also noted that if a garment contains several pictures, 
  it would be treated as several garments of the same text labels and scores.

And finally the binary vector constructed from text labels would be concatenated to another vector which describes image features to represent the whole features.

For every user, he or she will judge a garment based on personal 
  preference with scores from 1 to 5. 
All the choices for clothing are recorded for future recommendation. 
Then based on the certain user's previous choices, the potential 
  high-score garments can be predicted and thus being recommended.

Once we get the user's preferences (scores) on some garments, 
  the relationships between these garments and their scores on this
  particular user have been established, which are serving as a training set. 
Then we use SVM to estimate the scores of other garments, based on which, 
  we pick the highest ones to recommend. 
Note that this whole process is based on this particular user's preference.
There's also another optional way to recommend. 
The garments we recommend will have a high possibility of user's preference, 
  which is derived from the relationship between the preference models of different users.

Formally, suppose there are $K$ keywords in total, then every item $I$ has an associated 
  $K$-dimensional vector $T(I)$ where each dimension is either 0 or 1 which indicates the 
  existence of the corresponding keyword in the text labels of item $I$.
For an item $I$ and a user $U$, let $s(U,I)$ denote the score of $I$ given by $U$. 
Let $M(I)$ be the collection of all the images of $I$, $D$ be the learnt dictionary, and
  $C(M,D)$ be the sparse coding of $M$ using $D$.
Then the training set can be represented as $$ \{ [C(M^*,D), T(I), s(U,I)] : M^* \in M(I) \} .$$
