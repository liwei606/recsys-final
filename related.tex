\section{Related Work}\label{sec:related}

Collaborative Filtering (CF) \cite{Goldberg92} has become a very common technique 
  for providing personalized recommendations, suggesting items based on the similarity 
  between users' preferences.
Sarwar \etal\cite{Sarwar01} analyzed different item-based recommendation generation algorithms 
  and stated that item-based algorithms provide dramatically better performance than 
  user-based algorithms.

Zhang \etal\cite{Zhang10} proposed an integrated diffusion-based algorithm, which is very much 
  different from collaborative filtering, with the help of collaborative tagging 
  information, and experimental results demonstrated significant improvement in accuracy, 
  diversification and novelty of recommendations. 
However, it is not an online algorithm and cannot give real-time response to the user activities.

WebCF-DT \cite{Kim02} is a personalized recommendation procedure based on Web usage mining, 
  product taxonomy, association rule mining, and decision tree induction. 
It relies heavily on behavior patterns and lacks understanding of recommended materials. 
Also they don't have a comparison with collaborative filtering and rule-based approaches, 
  therefore the effectiveness is not convincing.

INTRIGUE \cite{Ardissono03} is a prototype tourist-information server that recommends 
  sightseeing destinations and itineraries by taking into account the preferences of 
  heterogeneous tourist groups instead of individuals.
INTRIGUE uses a na\"ive polynomial formula for generating scores of items and thus has less 
  intelligence than approaches supported by machine learning techniques.
