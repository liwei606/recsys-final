\section{Technical Challenges}\label{sec:challenge}

In this section, several technical challenges in personalized apparel recommendation 
  are listed and discussed with respect to our framework. 

\subsection{How to Model an Item}
An \emph{item} means a garment along with several images and text labels.
For personalized apparel recommendation, one common way is to extract 
  the text labels (sizes, styles, colors, materials, etc.) and do recommendation 
  based on the statistics of and the relationships among these labels.
However, it ignores lots of information provided by the images of the garments.
As stated in Section \ref{sec:intro}, personalized apparel recommendation is 
  very much sensitive in that the choices of people with similar interests 
  may differ in case of only tiny differences. 
One of the novelties in our work is to model items by using the image features.
For image features, we strive to capture the most distinguishing or significant ones 
  in order to improve the sensitivity of our recommendation.

\subsection{How to Model User Preference}
In E-commerce systems, various actions can be captured and classified as user preferences,
  for example, viewing item page, checking details (mouse clicking), adding to cart, adding to favorites.
However, these kinds of actions are relatively easy to trigger and have less side effects.
So there tends to be many of these actions which may include noises and cause misleading recommendations.
In our framework, the users are required to judge a garment with scores based on personal preference,
  i.e. to give ratings manually.
Along with the scores, the users can also specify the confidence values of the rating. 
For those easy-to-trigger actions (or events), it is trivial to generate scores with appropriate 
  confidence values on behalf of the users.

\subsection{How to Recommend}
Given the item models and user preference models, the next problem to solve is how to recommend.
One possible way is to use a na\"ive Bayes classifier or similar probabilistic classifiers.
However, these kinds of classifiers require some extent of distribution independence which may not, 
  or relatively hard to, be satisfied under the setting of the item models and preference models. 
In our framework, the item models are required to be in the form of vectors, and an SVM classifier 
  is trained from user ratings of everybody and used as the preference model for each user.
Eventually there will be one SVM classifier for each user and the framework recommends according to
  the scores given by the classifier. 

\subsection{How to Recommend in an Online Context}
For any of the item modeling, the user preference modeling and the recommendation, 
  it should be done in an \emph{online} context because of the nature of E-commerce systems.
This means most of the batch-mode offline algorithms can not be applied in the framework.
Since this involves more fundamental aspects of image processing and machine learning, 
  instead of devising new techniques or reinventing the wheels, we examine and adopt several 
  suitable algorithms in our framework for further selection. 
